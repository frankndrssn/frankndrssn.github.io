\documentclass[oneside,a4paper]{book}\usepackage[final]{microtype}\usepackage{amsthm,mathtools}\usepackage{xcolor}\usepackage[colorlinks=true,linkcolor={blue!30!black}]{hyperref}\newtheorem{theorem}{Theorem}[chapter]\newtheorem{lemma}[theorem]{Lemma}\newtheorem{observation}[theorem]{Observation}\newtheorem{axiom}[theorem]{Axiom}\newtheorem{corollary}[theorem]{Corollary}\theoremstyle{definition}\newtheorem{definition}[theorem]{Definition}\newtheorem{construction}[theorem]{Construction}\newtheorem{example}[theorem]{Example}\newtheorem{convention}[theorem]{Convention}\newtheorem{exercise}{Exercise}\usepackage{newtxmath,newtxtext}\usepackage[mode=buildmissing]{standalone}\setcounter{tocdepth}{5}\setcounter{secnumdepth}{5}\title{Explicit description of the subobject classifier}\author{}\begin{document}
\begin{filecontents*}[overwrite]{4ceec40ab734eb482ace78ab6f078755.tex}
\documentclass[crop]{standalone}

         
   \RequirePackage {tikz}
   \RequirePackage {amsmath}
   \usetikzlibrary {backgrounds, intersections, calc, spath3, fit}

   \definecolor {catccolor}{RGB}{255,244,138}
   \tikzstyle {dot}=[circle, draw=black, fill=black, minimum size=1mm, inner sep=0mm]
   \tikzstyle {catc}=[catccolor!60]
   \tikzstyle {catd}=[orange!40]
   \tikzstyle {cate}=[red!40]
   \tikzstyle {catf}=[blue!10]
   \tikzstyle {catg}=[green!25]


   \tikzstyle {blue halo}=[fill=blue!10, opacity=0.7, rounded corners]
   \tikzstyle {white halo}=[fill=white, opacity=0.7, rounded corners]

   \NewDocumentCommand \CreateRect {D<>{} m m}{
     \path 
      coordinate (#1nw)
      ++(#2,-#3) coordinate (#1se)
      coordinate (#1sw) at (#1se -| #1nw)
      coordinate (#1ne) at (#1nw -| #1se)
    ;

     \path [spath/save = #1north] (#1nw) to (#1ne);
     \path [spath/save = #1west] (#1nw) to (#1sw);
     \path [spath/save = #1east] (#1ne) to (#1se);
     \path [spath/save = #1south] (#1sw) to (#1se);
  }
 
         
   % appears to be necessary
  \usepackage{tikz, tikz-cd, mathtools, amssymb, stmaryrd}
  \usetikzlibrary{matrix,arrows}
  \usetikzlibrary{backgrounds,fit,positioning,calc,shapes}
  \usetikzlibrary{decorations.pathreplacing}
  \usetikzlibrary{decorations.pathmorphing}
  \usetikzlibrary{decorations.markings}

  % basic settings
  \tikzset{
    diagram/.style = {
      on grid,
      node distance=2cm,
      commutative diagrams/every diagram,
      line width = .5pt,
      every node/.append style = {
        commutative diagrams/every cell,
      },
    },
    morphism/.style = {
      auto=left,
      node font=\scriptsize,
    },
    phantom/.style = {
      commutative diagrams/phantom,
      node font=\scriptsize,
      sloped,
    },
    between/.style args = {#1 and #2}{
      at = ($(#1)!0.5!(#2)$)
    },
    desc/.style = {anchor=center,fill=white,inner sep=3pt},
    offset/.style args = {#1} {commutative diagrams/shift right={#1}},
    cross over/.style = {
      commutative diagrams/crossing over
    },
  }

  % macro that draws the adjunction symbol in a diagram
  \NewDocumentCommand\DrawAdj{smm}{
    \IfBooleanTF {#1}{
      \draw [phantom] (#2) to node {$\dashv$} (#3);
    }{
      \draw [phantom] (#2) to node[rotate=180] {$\dashv$} (#3);
    }
  }

  % pullback and pushout
  \tikzset{
    pullback/.style = {
      append after command = {
        \pgfextra{
          \draw ($(\tikzlastnode) + (.2cm,-.5cm)$) -- ++(0.3cm,0) -- ++(0,0.3cm);
        }
      }
    },
    pushout/.style = {
      append after command={
        \pgfextra{
          \draw ($(\tikzlastnode) + (-.2cm,.5cm)$) -- ++(-0.3cm,0) -- ++(0,-0.3cm);
        }
      }
    },
  }

  % arrow styles
  \tikzset{
    exists/.style = {
      densely dashed
    },
    embedding/.style = {
      {right hook}->
    },
    cover/.style = {
      -{Triangle[open]}
    },
  }

  \newlength{\dontworryaboutit} % no idea what it does

  \tikzset{
    inline diagram/.style = {
      commutative diagrams/every diagram,
      commutative diagrams/cramped,
      line width = .5pt,
      every node/.append style = {
        commutative diagrams/every cell,
        anchor = base,
        inner sep = 0pt
      },
      every path/.append style = {
        outer xsep = 2pt
      }
    }
  }
  
  % macros for drawing commutative squares
  % adapted from Jon Sterling's macros (https://git.sr.ht/~jonsterling/article-template/tree/main/item/local-tikz.sty)
  \tikzset{
    square/nw/.style = {},
    square/ne/.style = {},
    square/se/.style = {},
    square/sw/.style = {},
    square/north/.style = {->},
    square/south/.style = {->},
    square/west/.style = {->},
    square/east/.style = {->},
    square/north/node/.style = {above},
    square/south/node/.style = {below},
    square/west/node/.style = {left},
    square/east/node/.style = {right},
  }

  \ExplSyntaxOn
  \keys_define:nn { jon-tikz/diagram } {
    nw .tl_set:N = \l_jon_tikz_diagram_nw,
    sw .tl_set:N = \l_jon_tikz_diagram_sw,
    ne .tl_set:N = \l_jon_tikz_diagram_ne,
    se .tl_set:N = \l_jon_tikz_diagram_se,

    width .tl_set:N = \l_jon_tikz_diagram_width,
    height .tl_set:N = \l_jon_tikz_diagram_height,

    north .tl_set:N = \l_jon_tikz_diagram_north,
    south .tl_set:N = \l_jon_tikz_diagram_south,
    west .tl_set:N = \l_jon_tikz_diagram_west,
    east .tl_set:N = \l_jon_tikz_diagram_east,

    nw/style .code:n = {\tikzset{square/nw/.style = {#1}}},
    sw/style .code:n = {\tikzset{square/sw/.style = {#1}}},
    ne/style .code:n = {\tikzset{square/ne/.style = {#1}}},
    se/style .code:n = {\tikzset{square/se/.style = {#1}}},

    north/style .code:n = {\tikzset{square/north/.style = {#1}}},
    north/node/style .code:n = {\tikzset{square/north/node/.style = {#1}}},
    south/style .code:n = {\tikzset{square/south/.style = {#1}}},
    south/node/style .code:n = {\tikzset{square/south/node/.style = {#1}}},
    west/style .code:n = {\tikzset{square/west/.style = {#1}}},
    west/node/style .code:n = {\tikzset{square/west/node/.style = {#1}}},
    east/style .code:n = {\tikzset{square/east/.style = {#1}}},
    east/node/style .code:n = {\tikzset{square/east/node/.style = {#1}}},

    draft .meta:n = {
      nw = {nw},
      sw = {sw},
      se = {se},
      ne = {ne},
      north = {north},
      south = {south},
      west = {west},
      east = {east},
    }
  }

  \tl_set:Nn \l_jon_tikz_diagram_width { 2cm }
  \tl_set:Nn \l_jon_tikz_diagram_height { 2cm }


  \cs_new:Nn \__jon_tikz_render_square:nn {
    \group_begin:
    \keys_set:nn {jon-tikz/diagram} {#2} {
      \node (#1nw) [square/nw] {$\l_jon_tikz_diagram_nw$};
      \node (#1sw) [below = \l_jon_tikz_diagram_height~of~#1nw,square/sw] {$\l_jon_tikz_diagram_sw$};
      \draw[morphism,square/west] (#1nw) to node [square/west/node] {$\l_jon_tikz_diagram_west$} (#1sw);

      \node (#1ne) [right = \l_jon_tikz_diagram_width~of~#1nw,square/ne] {$\l_jon_tikz_diagram_ne$};
      \node (#1se) [below = \l_jon_tikz_diagram_height~of~#1ne,square/se] {$\l_jon_tikz_diagram_se$};
      \draw[morphism,square/north] (#1nw) to node [square/north/node] {$\l_jon_tikz_diagram_north$} (#1ne);
      \draw[morphism,square/east] (#1ne) to node [square/east/node] {$\l_jon_tikz_diagram_east$} (#1se);
      \draw[morphism,square/south] (#1sw) to node [square/south/node] {$\l_jon_tikz_diagram_south$} (#1se);
    }
    \group_end:
  }

  % the optional argument allows one to specify a prefix to positions
  \NewDocumentCommand\SpliceDiagramSquare{O{}m}{
    \__jon_tikz_render_square:nn {#1} {#2}
  }

  \NewDocumentCommand\DiagramSquare{O{}O{}m}{
    \begin{tikzpicture}[diagram,#2,baseline=(#1sw.base)]
      \__jon_tikz_render_square:nn {#1} {#3}
    \end{tikzpicture}
  }
  \ExplSyntaxOff
 
      \usepackage{newtxmath,newtxtext}
\begin{document}
   \DiagramSquare {
    nw = A',
    nw/style = pullback,
    ne =  \Omega ',
    sw = A,
    se =  \Omega ,
    west = m,
    east =  \top ,
    west/style = {>->},
    east/style = {>->},
    south =  \chi _{m},
    south/style = {->,exists},
  }
\end{document}
\end{filecontents*}

\begin{filecontents*}[overwrite]{\jobname.bib}

\end{filecontents*}
\frontmatter\maketitle\tableofcontents\mainmatter\par{}Explicitly, a subobject classifier is an object \(\Omega\) equipped with a \emph{generic subobject} \(\top \colon   \Omega '  \rightarrowtail   \Omega\) such that for any subobject \(m \colon  A'  \rightarrowtail  A\), there is a unique \emph{\textbf{classifying morphism}} \(\chi _{m} \colon  A  \to   \Omega\) such that \(m\) is a pullback against \(\top\) along \(\chi _{m}\).\begin{center}\includestandalone{4ceec40ab734eb482ace78ab6f078755}\end{center}\backmatter\nocite{*}\bibliographystyle{plain}\bibliography{\jobname.bib}\end{document}